\chapter{Исследовательский раздел}

Наполним базу данных различными записями при помощи следующих функций:
\begin{itemize}
  \item 
    \verb|add_rams()| для заполнения таблицы RAM (листинг \ref{lst:add_rams.sql})
  \item 
    \verb|add_cpus()| для заполнения таблицы CPUs(листинг \ref{lst:add_cpus.sql})
  \item 
    \verb|add_gcards()| для заполнения таблицы GraphicCards(листинг \ref{lst:add_gcards.sql}), а также отдельно предварительного создания записей для GPUs
  \item 
    \verb|add_hdds()| для заполнения таблицы HDD(листинг \ref{lst:add_hdds.sql})
  \item 
    \verb|add_ssds()| для заполнения таблицы SSD(листинг \ref{lst:add_ssds.sql})
  \item 
    \verb|add_ssdsm2()| для заполнения таблицы SSDM2(листинг \ref{lst:add_ssdsm2.sql})
  \item 
    \verb|add_motherboards()| для заполнения таблицы Motherboards(листинг \ref{lst:add_motherboards.sql})
\end{itemize}

%     \includelistingpretty
%     {add_rams.sql} % Имя файла с расширением (файл должен быть расположен в директории inc/lst/)
%     {sql} % Язык программирования (необязательный аргумент)
%     {Функция добавления тестируемых записей в таблицу RAM} % Подпись листинга
% % \mylisting[sql]{function.sql}{firstline=1,lastline=112}{Функция поиска клубов по заданным параметрам}{function}{}

\listingminted{add_rams.sql}{sql}{Функция добавления тестируемых записей в таблицу RAM}

\listingminted{add_cpus.sql}{sql}
{Функция добавления тестируемых записей в таблицу CPUs}

\listingminted{add_gcards.sql}{sql}
{Функция добавления тестируемых записей в таблицу GraphicCards}

\listingminted{add_hdds.sql}{sql}
{Функция добавления тестируемых записей в таблицу HDD}

\listingminted{add_ssds.sql}{sql}
{Функция добавления тестируемых записей в таблицу SSD}

\listingminted{add_ssdsm2.sql}{sql}
{Функция добавления тестируемых записей в таблицу SSDM2}

\listingminted{add_motherboards.sql}{sql}
{Функция добавления тестируемых записей в таблицу Motherboards}

Выдача прав пользователю для работы с данными в таблице выполняется при помощи интерпретатора в терминале psql от пользователя с правами администратора, подключенным к базе данных perscompcomps (листинг \ref{lst:grants.sql})
\listingminted{grants.sql}{sql}
{Добавление прав для работы с данными базы данных perscompcomps}

Результаты добавления данных на рисунках 
\ref{img:add_rams}, 
\ref{img:add_cpus}, 
\ref{img:add_gcards}, 
\ref{img:add_hdds}, 
\ref{img:add_ssds}, 
\ref{img:add_ssdsm2}, 
\ref{img:add_motherboards}.

\includeimage
{add_rams} % Имя файла без расширения (файл должен быть расположен в директории inc/img/)
{f} % Обтекание (без обтекания)
{h} % Положение рисунка (см. figure из пакета float)
{\textwidth} % Ширина рисунка
{Отображение результата выполненного запроса функции add\_rams} % Подпись рисунка

\includeimage
{add_cpus} % Имя файла без расширения (файл должен быть расположен в директории inc/img/)
{f} % Обтекание (без обтекания)
{h} % Положение рисунка (см. figure из пакета float)
{\textwidth} % Ширина рисунка
{Отображение результата выполненного запроса функции add\_cpus} % Подпись рисунка

\includeimage
{add_gcards} % Имя файла без расширения (файл должен быть расположен в директории inc/img/)
{f} % Обтекание (без обтекания)
{h} % Положение рисунка (см. figure из пакета float)
{\textwidth} % Ширина рисунка
{Отображение результата выполненного запроса функции add\_gcards} % Подпись рисунка

\includeimage
{add_hdds} % Имя файла без расширения (файл должен быть расположен в директории inc/img/)
{f} % Обтекание (без обтекания)
{h} % Положение рисунка (см. figure из пакета float)
{\textwidth} % Ширина рисунка
{Отображение результата выполненного запроса функции add\_hdds} % Подпись рисунка

\includeimage
{add_ssds} % Имя файла без расширения (файл должен быть расположен в директории inc/img/)
{f} % Обтекание (без обтекания)
{h} % Положение рисунка (см. figure из пакета float)
{\textwidth} % Ширина рисунка
{Отображение результата выполненного запроса функции add\_ssds} % Подпись рисунка

\includeimage
{add_ssdsm2} % Имя файла без расширения (файл должен быть расположен в директории inc/img/)
{f} % Обтекание (без обтекания)
{h} % Положение рисунка (см. figure из пакета float)
{\textwidth} % Ширина рисунка
{Отображение результата выполненного запроса функции add\_ssdsm2} % Подпись рисунка

\includeimage
{add_motherboards} % Имя файла без расширения (файл должен быть расположен в директории inc/img/)
{f} % Обтекание (без обтекания)
{h} % Положение рисунка (см. figure из пакета float)
{\textwidth} % Ширина рисунка
{Отображение результата выполненного запроса функции add\_motherboards} % Подпись рисунка

Все запросы выполнены успешны. Выполним запрос на получение данных из всех таблиц.

\includeimage
{select_rams} % Имя файла без расширения (файл должен быть расположен в директории inc/img/)
{f} % Обтекание (без обтекания)
{h} % Положение рисунка (см. figure из пакета float)
{\textwidth} % Ширина рисунка
{Отображение результата выполненного запроса получения данных из таблицы rams с объединением данных от наследуемой таблицы components} % Подпись рисунка

\includeimage
{select_cpus} % Имя файла без расширения (файл должен быть расположен в директории inc/img/)
{f} % Обтекание (без обтекания)
{h} % Положение рисунка (см. figure из пакета float)
{\textwidth} % Ширина рисунка
{Отображение результата выполненного запроса получения данных из таблицы cpus с объединением данных от наследуемой таблицы components} % Подпись рисунка

\includeimage
{select_gcards} % Имя файла без расширения (файл должен быть расположен в директории inc/img/)
{f} % Обтекание (без обтекания)
{h} % Положение рисунка (см. figure из пакета float)
{\textwidth} % Ширина рисунка
{Отображение результата выполненного запроса получения данных из таблицы graphiccards с объединением данных от наследуемой таблицы components} % Подпись рисунка

\includeimage
{select_hdds} % Имя файла без расширения (файл должен быть расположен в директории inc/img/)
{f} % Обтекание (без обтекания)
{h} % Положение рисунка (см. figure из пакета float)
{\textwidth} % Ширина рисунка
{Отображение результата выполненного запроса получения данных из таблицы hdd с объединением данных от наследуемой таблицы components} % Подпись рисунка

\includeimage
{select_ssds} % Имя файла без расширения (файл должен быть расположен в директории inc/img/)
{f} % Обтекание (без обтекания)
{h} % Положение рисунка (см. figure из пакета float)
{\textwidth} % Ширина рисунка
{Отображение результата выполненного запроса получения данных из таблицы ssd с объединением данных от наследуемой таблицы components} % Подпись рисунка

На рисунке \ref{img:select_ssds} видим, что данных больше чем было добавлено в таблицу ssd, что верно. Остальные данные связаны с данными таблицы ssdm2, которая является наследуемой от ssd. На рисунке \ref{img:select_ssdsm2} подтверждение этому --- записи таблицы ssdm2 действительно унаследованы от ssd и связаны по внешнему ключу. 

\includeimage
{select_ssdsm2} % Имя файла без расширения (файл должен быть расположен в директории inc/img/)
{f} % Обтекание (без обтекания)
{h} % Положение рисунка (см. figure из пакета float)
{\textwidth} % Ширина рисунка
{Отображение результата выполненного запроса получения данных из таблицы ssdm2 с объединением данных от наследуемой таблицы components} % Подпись рисунка

На рисунке \ref{img:select_nvm} отображен результат получения всех данных множества энергонезависимой памяти благодаря наследованию атрибутов таблицы nonvolatilemem.

\includeimage
{select_nvm} % Имя файла без расширения (файл должен быть расположен в директории inc/img/)
{f} % Обтекание (без обтекания)
{h} % Положение рисунка (см. figure из пакета float)
{\textwidth} % Ширина рисунка
{Отображение результата выполненного запроса получения данных из таблицы nonvolatilemem с объединением данных от наследуемой таблицы components} % Подпись рисунка

\includeimage
{select_motherboards} % Имя файла без расширения (файл должен быть расположен в директории inc/img/)
{f} % Обтекание (без обтекания)
{h} % Положение рисунка (см. figure из пакета float)
{\textwidth} % Ширина рисунка
{Отображение результата выполненного запроса получения данных из таблицы motherboards с объединением данных от наследуемой таблицы components} % Подпись рисунка

На рисунке \ref{img:select_components} отображен результат запроса всех записей из таблицы components. Все записи множества компонентов наследуемых таблиц от components имеются

\includeimage
{select_components} % Имя файла без расширения (файл должен быть расположен в директории inc/img/)
{f} % Обтекание (без обтекания)
{h} % Положение рисунка (см. figure из пакета float)
{\textwidth} % Ширина рисунка
{Отображение результата выполненного запроса получения данных из таблицы components} % Подпись рисунка

Убедимся, что функции выполнены верно, для чего выполним запросы получения данных из таблиц стран (рисунок \ref{img:select_countries}), типов оперативной памяти (рисунок \ref{img:select_ddrtypes}), производителей (рисунок \ref{img:select_manufacturers}), сокетов процессоров (рисунок \ref{img:select_cpusockets}), графических процессоров (рисунок \ref{img:select_gpus})

\includeimage
{select_countries} % Имя файла без расширения (файл должен быть расположен в директории inc/img/)
{f} % Обтекание (без обтекания)
{h} % Положение рисунка (см. figure из пакета float)
{\textwidth} % Ширина рисунка
{Отображение результата выполненного запроса получения данных из таблицы countries} % Подпись рисунка

\includeimage
{select_ddrtypes} % Имя файла без расширения (файл должен быть расположен в директории inc/img/)
{f} % Обтекание (без обтекания)
{h} % Положение рисунка (см. figure из пакета float)
{\textwidth} % Ширина рисунка
{Отображение результата выполненного запроса получения данных из таблицы ddrtypes} % Подпись рисунка

\includeimage
{select_manufacturers} % Имя файла без расширения (файл должен быть расположен в директории inc/img/)
{f} % Обтекание (без обтекания)
{h} % Положение рисунка (см. figure из пакета float)
{\textwidth} % Ширина рисунка
{Отображение результата выполненного запроса получения данных из таблицы manufacturers} % Подпись рисунка

\includeimage
{select_cpusockets} % Имя файла без расширения (файл должен быть расположен в директории inc/img/)
{f} % Обтекание (без обтекания)
{h} % Положение рисунка (см. figure из пакета float)
{\textwidth} % Ширина рисунка
{Отображение результата выполненного запроса получения данных из таблицы cpusockets} % Подпись рисунка

\includeimage
{select_gpus} % Имя файла без расширения (файл должен быть расположен в директории inc/img/)
{f} % Обтекание (без обтекания)
{h} % Положение рисунка (см. figure из пакета float)
{\textwidth} % Ширина рисунка
{Отображение результата выполненного запроса получения данных из таблицы gpus} % Подпись рисунка

Видим, что данные добавлены, не продублированы, были условно добавлены при остутствии или найдены при использовании реализованых функций.

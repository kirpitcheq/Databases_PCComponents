\usepackage{longtable}
\usepackage{array}
\usepackage{booktabs}
\usepackage{longtable}
\usepackage{tabularx}
\usepackage{makecell}

\usepackage{datetime}
\usepackage{soul}
\usepackage{xstring}

% set enumerate
\renewcommand{\labelenumii}{\arabic{enumi}.\arabic{enumii}}
\renewcommand{\labelenumiii}{\arabic{enumi}.\arabic{enumii}.\arabic{enumiii}}
\renewcommand{\labelenumiv}{\arabic{enumi}.\arabic{enumii}.\arabic{enumiii}.\arabic{enumiv}}

\newcommand{\bmstuY}{\the\year}
\newcommand{\bmstuYear}{\bmstuY{} г.}
\newcommand{\bmstuCityYear}{г. Москва, \bmstuYear{}}

\newcommand{\headerSize}{\small}
\newcommand{\bmstuHeader}{
  \centering
  {
	  \singlespacing \small
	  Министерство науки и высшего образования Российской Федерации \\
	  Федеральное государственное бюджетное образовательное учреждение \\
	  высшего образования \\
	  <<Московский государственный технический университет \\
	  имени Н.~Э.~Баумана \\
	  (национальный исследовательский университет)>> \\
	  (МГТУ им. Н.~Э.~Баумана) \\
  }
  \vspace{-4.2mm}
  \vhrulefill{0.9mm} \\
  \vspace{-7mm}
  \vhrulefill{0.2mm} \\
  }

  \newcommand{\phantomBmstuTaskYear}{20\phantom{\bmstuYear{}} г.}
  \newcommand{\bmstuDateRule}{\underline{«\the\day» \monthname[\the\month] \bmstuYear{}}}
  \newcommand{\bmstuPhantomDateRule}{\underline{«\phantom{\the\day}» \phantom{\monthname[\the\month]} \phantom{\bmstuYear{}}}}
%--------------------------------------
  \newcommand{\makeDepartHeadStatement}[2]{
    \begin{flushright}
      {
	      \singlespacing\taskTextSize\noindent
	      \begin{tabular}{@{}cc@{}}
          УТВЕРЖДАЮ &\\
          Заведующий кафедрой & \underline{#1}\\
          & \tiny(Индекс) \\
          \underline{\phantom{Заведующий кафедрой}}& \underline{#2} \\
          & \tiny(И.О. Фамилия) \\
          \bmstuPhantomDateRule{} & \underline{\phantomBmstuTaskYear{}} \\
        \end{tabular}
      }
    \end{flushright}
  }
  %--------------------------------------
  \newcommand{\makelargetask}[1]{
    \begin{center}
      {
	      \singlespacing \large
        {\textbf{ЗАДАНИЕ}}\\
        {\textbf{на выполнение #1}}
      }
    \end{center}
  }
  %--------------------------------------
  \newcommand{\taskpoints}[1]{
    {
	    \singlespacing \footnotesize
	    \setlist{nosep}
	    \setlist{nolistsep}
      \begin{enumerate}[leftmargin=*]
        \item 
          \textbf{Техническое задание:} \\
          \noindent\bmstuTechTask{} \\
        \item 
          \textbf{Оформление #1:}
          \begin{enumerate}[leftmargin=*]
            \item
              Расчетно-пояснительная записка на 25-30 листах формата А4. \\
              Расчетно-пояснительная записка должна содержать постановку задачи, введение, аналитическую часть, конструкторскую часть, технологическую часть, исследовательский раздел, заключение, список литературы. 
            \item
              Перечень графического материала (плакаты, схемы, чертежи и т.п.). На защиту проекта должна быть представлена презентация, состоящая из 15-20 слайдов. На слайдах должны быть отражены: постановка задачи, использованные методы и алгоритмы, расчетные соотношения, структура комплекса программ, диаграмма классов, интерфейс, характеристики разработанного ПО, результаты проведенных исследований.
          \end{enumerate}

      \end{enumerate}
      Дата выдачи задания \bmstuPhantomDateRule{}\underline{\phantomBmstuTaskYear{}}
    }
  }
  %--------------------------------------
    \newcommand{\makefields}[3]{
      \vspace*{\fill}
      \noindent\begin{tabularx}{\textwidth}{@{}lcXcl@{}}
        Студент & \underline{#1} & &\underline{\phantom{(Подпись, дата)}} & \underline{#2} \\
        & \tiny(Группа) & &\tiny(Подпись, дата) & \tiny(И.О. Фамилия) \\
        \\
        Руководитель~курсовой~работы & & &\underline{\phantom{(Подпись, дата)}} & \underline{#3} \\
        & & &\tiny(Подпись, дата) & \tiny(И.О. Фамилия) \\
      \end{tabularx}
    }
  %--------------------------------------
      \newcommand{\maketaskdesc}[5]{
      \noindent\textbf{по дисциплине: }\underline{#1}
      \\Студент группы \underline{#2}\\
      \underline{#3} \\
      {\tiny(Фамилия, имя, отчество)}\\
      Тема #4: #5 \\
      Направленность работы (учебная, исследовательская, практическая, производственная, др.) \underline{учебная} \\
      Источник тематики (кафедра, предприятие, НИР) \underline{кафедра} \\
      График выполнения работы: 25\% к \underline{4} нед., 50\% к \underline{7} нед., 75\% к \underline{11} нед., 100\% к \underline{14} нед.
  }
  %--------------------------------------

  \newcommand{\makecourseworktask}[9]
  {
    \newcommand{\bmstuDepart}{#1}
    \newcommand{\headOfDepart}{#2}
    \newcommand{\bmstuDiscipl}{#3}
    \newcommand{\bmstuGroup}{#4}
    \newcommand{\bmstuFullStudName}{#5}
    \newcommand{\bmstuTheme}{#6}
    \newcommand{\bmstuTechTask}{#7}
    \newcommand{\bmstuStudName}{#8}
    \newcommand{\bmstuWorkHead}{#9}

    \newcommand{\bmstuTaskType}{курсовой работы}
    \newcommand{\taskTextSize}{\footnotesize}
    \newpage
    {
      \bmstuHeader
    }
    \makeDepartHeadStatement{\bmstuDepart}{\headOfDepart}

    \makelargetask{\bmstuTaskType}

    {
	    \singlespacing \taskTextSize
      \maketaskdesc{\bmstuDiscipl}{\bmstuGroup}{\bmstuFullStudName}{\bmstuTaskType}{\bmstuTheme}

      \taskpoints{\bmstuTaskType}


      \vfill

		  \noindent\begin{tabularx}{\textwidth}{@{}>{\hsize=.5\hsize}X>{\hsize=.25\hsize}X>{\hsize=.25\hsize}X@{}}
			  \titlepagestudentscontent

			  \titlepageotherscontent
		  \end{tabularx}
      % \maketitlepageothers{}
      % \makefields{\bmstuGroup}{\bmstuStudName}{\bmstuWorkHead}
    }
  }


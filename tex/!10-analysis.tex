\chapter{Аналитический раздел}

\section{Анализ предметной области}
Обычный персональный компьютер состоит из системного блока, состоящего из шасси и периферийных устройств.

Материнская (системная) плата --- печатная плата, являющаяся основой построения модульного персонального компьютера.
Системная плата содержит основную часть персонального компьютера.
На материнской плате установлен чипсет --- взаимосвязанный набор микросхем, логика которого в наибольшей степени определяет то, каковы прочие элементы компьютера и с каким типом устройств может работать данный чипсет, например какой тип стандарта оперативной памяти DDR поддерживает, с каким типом процессоров работает, какой тип PCI шины использует и т.д.

Центральный процессор --- интегральная схема, исполняющая машинные инструкции (код программ), главная часть аппаратного обеспечения компьютера или программируемого логического контроллера. 

Оперативная память (англ. Random Access Memory, RAM --- память с произвольным доступом) --- в большинстве случаев энергозависимая часть системы компьютерной памяти, в которой во время работы компьютера хранится выполняемый машинный код (программы), а также входные, выходные и промежуточные данные, обрабатываемые процессором. Оперативное запоминающее устройство (ОЗУ) --- техническое устройство, реализующее функции оперативной памяти
Для повышения надёжности, облегчения модернизации и экономии места на материнской плате микросхемы ОЗУ объединяют в модули, которые устанавливаются на плату вертикально.
Наиболее распространенный стандарт оперативной памяти --- DDR. Современные устройства еще используют версии DDR3, все больше DDR4, некоторые уже DDR5. Все типы являются несовместимыми и отличаются слотом подключения. 

Видеокарта (видеоадаптер, графический адаптер) --- устройство, преобразующее графический образ, хранящийся как содержимое памяти компьютера (или самого адаптера), в форму, пригодную для дальнейшего вывода на экран монитора. Обычно видеокарта выполнена в виде печатной платы (плата расширения) и вставляется в слот расширения материнской платы --- универсальный, либо специализированный (AGP, PCI Express).
Также широко распространены и расположенные на системной плате видеокарты --- как в виде дискретного отдельного чипа GPU, так и в качестве составляющей части северного моста чипсета или ЦПУ; в случае ЦПУ, встроенный (интегрированный) GPU, строго говоря, не может быть назван видеокартой.
Видеокарты не ограничиваются простым выводом изображения. Они имеют встроенный графический процессор, который может производить дополнительную обработку, снимая эту задачу с центрального процессора компьютера. Например, видеокарты Nvidia и AMD (ATi) осуществляют рендеринг графического конвейера OpenGL, DirectX и Vulkan на аппаратном уровне.
Также имеет место тенденция использовать вычислительные возможности графического процессора для решения неграфических задач (например, добычи криптовалюты или параллельных вычислений, таких как BOINC).

Энергонезависимая память (англ. Non-Volatile Random-Access Memory; NVRAM) --- разновидность запоминающих устройств с произвольным доступом, которые способны хранить данные при отсутствии электрического питания.
К энергонезависимым носителям: жесткие диски(HDD), твердотельные диски (SSD), флеш-накопители, лазерные диски и т.д.

Жёсткий диск (hard disk drive (HDD)) - запоминающее устройство (устройство хранения информации, накопитель) произвольного доступа, основанное на принципе магнитной записи. Является основным накопителем данных в большинстве компьютеров.
Информация записывается на жёсткие (алюминиевые или стеклянные) пластины, покрытые слоем ферромагнитного материала, чаще всего диоксида хрома - магнитные диски. В HDD используется одна или несколько пластин на одной оси. Считывающие головки в рабочем режиме не касаются поверхности пластин благодаря прослойке набегающего потока воздуха, образующейся у поверхности при быстром вращении. Расстояние между головкой и диском составляет несколько нанометров (в современных дисках около 10 нм), а отсутствие механического контакта обеспечивает долгий срок службы устройства. При отсутствии вращения дисков головки находятся у шпинделя или за пределами диска в безопасной («парковочной») зоне, где исключён их нештатный контакт с поверхностью дисков.
Также, в отличие от гибкого диска, носитель информации обычно совмещают с накопителем, приводом и блоком электроники. Такие жёсткие диски часто используются в качестве несъёмного носителя информации.
Существуют несколько различных технологий записи: 
\begin{itemize}
    \item 
        Метод продольной записи --- технология CMR (англ. Conventional Magnetic Recording) --- это «обычная» магнитная запись, биты информации записываются с помощью маленькой головки, которая, проходя над поверхностью вращающегося диска, намагничивает миллиарды горизонтальных дискретных областей --- доменов. При этом вектор намагниченности домена расположен продольно, то есть параллельно поверхности диска. Каждая из этих областей является логическим нулём или единицей, в зависимости от направления намагниченности. 
    \item 
        Метод перпендикулярной записи --- технология PMR (англ. Perpendicular Magnetic Recording), при которой биты информации сохраняются в вертикальных доменах. Это позволяет использовать более сильные магнитные поля и снизить площадь материала, необходимую для записи 1 бита. Предыдущий метод записи, параллельно поверхности магнитной пластины, привёл к тому, что в определённый момент увеличивать плотность информации на дисках было невозможно. 
        Плотность записи при этом методе резко возросла - более чем на 30 \% ещё на первых образцах
    \item 
        Метод черепичной магнитной записи (англ. Shingled Magnetic Recording, SMR) был реализован в начале 2010-х годов. В нём используется тот факт, что ширина области чтения меньше, чем ширина записывающей головки. Запись дорожек в этом методе производится с частичным наложением в рамках групп дорожек (пакетов). Каждая следующая дорожка пакета частично закрывает предыдущую (подобно черепичной кровле), оставляя от неё узкую часть, достаточную для считывающей головки. По своей специфике она радикально отличается от более популярных технологий записи CMR и PMR.
        Черепичная запись увеличивает плотность записанной информации, однако осложняет перезапись - при каждом изменении требуется полностью перезаписать весь пакет перекрывающихся дорожек. Технология позволяет увеличить ёмкость жёстких дисков на 15-20 \% в зависимости от конкретной реализации; при этом не лишена недостатков, главный из которых - низкая скорость записи\/перезаписи, что критично при использовании в настольных компьютерах, поэтому применяется в основном в центрах обработки данных (ЦОД)
\end{itemize}
Со второй половины 2000-х годов получили распространение более производительные твердотельные накопители, вытесняющие дисковые накопители из ряда применений несмотря на более высокую стоимость единицы хранения; 
Твердотельный накопитель (англ. Solid-State Drive, SSD) - компьютерное энергонезависимое немеханическое запоминающее устройство на основе микросхем памяти, альтернатива жёстким дискам (HDD). Наиболее распространённый вид твердотельных накопителей использует для хранения данных флеш-память типа NAND(flash memory - разновидность полупроводниковой технологии электрически перепрограммируемой памяти (EEPROM) ). Помимо микросхем памяти, накопитель содержит управляющую микросхему - контроллер 
На 2016 год наиболее производительными выступали SSD формата M.2 с интерфейсом NVMe(NVM Express --- от англ. Non-Volatile Memory) --- интерфейс доступа к твердотельным накопителям, подключённым по шине PCI Express.), а к 2022 году их скорость записи\/чтения данных достигла 12000 мегабайт в секунду.
По сравнению с традиционными жёсткими дисками твердотельные накопители имеют меньший размер и вес, являются бесшумными, а также многократно более устойчивы к повреждениям (например, при падении) и имеют гораздо бóльшую скорость производимых операций. В то же время, они имеют в несколько раз бóльшую стоимость в пересчёте на гигабайт и меньшую износостойкость (ресурс записи).

\section{Формализация данных}
В состав системного блока входят:
\begin{itemize}
    \item 
        Материнская плата в которую устанавливается или на которой имеются:
        \begin{itemize}
            \item 
                чипсет, определяющий тип используемого процессора, оперативной памяти, шины и т.д.
            \item 
                процессор, 
            \item 
                оперативная память 
            \item 
                «встроенные» контроллеры периферийных устройств, и разъёмы для подключения дополнительных взаимозаменяемых плат, называемых платами расширений, как правило подключённые к общей шине, такие как 
                \begin{itemize}
                    \item 
                        видеокарта, 
                    \item 
                        аудиокарта, 
                    \item 
                        сетевая карта и т.д.
                \end{itemize}
                , разъемы для подключения устройств энергонезависимой постоянной памяти:
                \begin{itemize}
                    \item 
                        жёсткий диск (их может быть несколько, они могут быть объединены в RAID-массив)
                    \item 
                        SSD, SSD m2
                \end{itemize}
                , прочие разъемы для подключения устройств ввода-вывода:
                \begin{itemize}
                    \item 
                        usb различных стандартов: 2.0, 3.1, type-c и т.д.
                    \item 
                        видеоразъемы: hdmi, display-port, VGA
                    \item 
                        RJ-45 для подключения к сети ethernet
                    \item 
                        разъемы для подключения устройств ввода-вывода звука: для аналоговых сигналов и цифровых (S\/PDIF)
                    \item 
                        разъемы подключения устаревших устройств ввода-вывода (мышь, клавиатура) PS\/2
                    \item
                        COM-порт (последовательный порт), LTP (параллельный порт)
                    \item
                        и т.д.
                \end{itemize}
                (некоторые разъемы могут устанавливаться на плату с внутренней стороны, например COM-порт или USB)
            \item 
                ЦПУ.
                Главными характеристиками ЦПУ являются: 
                \begin{itemize}
                    \item 
                        тактовая частота, 
                    \item 
                        энергопотребление, 
                    \item 
                        нормы литографического процесса, используемого при производстве (для микропроцессоров), 
                    \item 
                        архитектура
                \end{itemize}
                Также ЦПУ Может иметь встроенный графический процессор ГПУ
            \item 
                ОЗУ
                Существуют несколько основных форматов плат ОЗУ:
                \begin{itemize}
                    \item 
                        DIMM(Dual In-line Memory Module). 
                    \item 
                        SO-DIMM (small outline - DIMM) - платы , используемые преимущественно на системных платах ноутбуков
                \end{itemize}

            \item
               Основными характеристиками графического процессора видеокарт являются: 
               \begin{itemize}
                   \item 
                       тактовая частота, 
                   \item 
                       энергопотребление, 
                   \item 
                       нормы литографического процесса, используемого при производстве (для микропроцессоров), 
                   \item 
                       архитектура
               \end{itemize}
               Характеристики видеокарты:
               \begin{itemize}
                   \item 
                       видеопамять:
                       \begin{itemize}
                           \item 
                               объем (также как и для оперативной памяти самым распространенным сейчас используют GDDR5),
                           \item 
                               памяти
                           \item 
                               пропускная способность
                           \item
                               разрядность шины
                       \end{itemize}
                   \item 
                       разъемы подключения
               \end{itemize}
        \end{itemize}
    \end{itemize}


\section{Модели баз данных}

База данных - это упорядоченный набор структурированной информации или данных, которые обычно хранятся в электронном виде в компьютерной системе. База данных обычно управляется системой управления базами данных (СУБД). Данные вместе с СУБД, а также приложения, которые с ними связаны, называются системой баз данных, или, для краткости, просто базой данных.

Данные в наиболее распространенных типах современных баз данных обычно хранятся в виде строк и столбцов формирующих таблицу. Этими данными можно легко управлять, изменять, обновлять, контролировать и упорядочивать. В большинстве баз данных для записи и запросов данных используется язык структурированных запросов (SQL). \cite{web:oracledef}

Рассмотрим основные типы использующихся различных типов баз данных:

\begin{itemize}
    \item
        Документо-ориентированные базы данных
    \item
        Ключ-значение
    \item
        Поисковые 
    \item 
        Реляционные
\end{itemize}

Документо-ориентированные базы данных в основном используются для хранения и обработки данных в формате документов, таких как JSON или XML. Они обеспечивают гибкость в структуре данных и удобны для хранения неструктурированных данных, позволяют легко добавлять новые, но ограничены в возможности выполнять сложные запросы данных, а также не позволяют связать большое количество данных.

Базы данных типа ключ-значение просты, т.к. для каждого значения имеется уникальный ключ, что облегчает доступ к данным, а также высокопроизводительны, однако такие хранилища обычно используют для кэширования временных данных, которые сохраняются в оперативной памяти, редко используются для сохранения данных на диск, в связи с чем такие системы практически и не применимы для сохранения на внешнее хранилище данных, из-за чего высок риск потери данных. Также системы не подходят для сложных связных структур данных.

Поисковые базы данных имеют мощные возможности для поиска и индексации данных и высокой производительностью при работе с текстом, но не с другими типами данных, а сложные структуры данных потребуют дополнительные затраты на обработку сложно связных структур данных.

Реляционные базы данных в отличии от остальных менее производительны, могут потребовать больше ресурсов, в зависимости от того как была спроектирована база данных, в связи с чем более сложны, но основаны на теории реляционных моделей данных, т.е. формализованы, обеспечивают целостность данных, позволяют создавать сложно связанные структуры данных, что допускает нормализацию данных для избежания их избыточности, предоставляют возможность резервировать данных. Такие типы систем зарекомендовали себя временем. Исходя из перечисленных достоинств и недостатков всех типов баз данных выбираем реляционные.

% это скорее в технологический
\section{СУБД}
СУБД --- это совокупность языковых и программных средств, предназначенных для создания, ведения и совместного использования БД многими пользователями. \cite{web:bseu}
Современная СУБД содержит в своем составе программные средства создания баз данных, средства работы с данными и сервисные средства. С помощью средств создания БД проектировщик, используя язык описания данных, переводит логическую модель БД в физическую структуру, а на языке манипуляции данными разрабатывает программы, реализующие основные операции с данными (в реляционных БД – это реляционные операции). При проектировании привлекаются визуальные средства, т.е. объекты, и программа-отладчик, с помощью которой соединяются и тестируются отдельные блоки разработанной программы управления конкретной БД.

Существует множество различных реляционных СУБД, самые распространенные из которых:
\begin{itemize}
    \item
        Microsoft Access.
        Разработана изначально под ОС Windows (имеется альтернатива LibreOffice Base, но не поддерживающая полный функционал и некоректно импортирует данные из MS Access)
    \item
        Oracle.
        Хорошо задокументирована, но весьма ограничена по функциональности для бесплатного использования, имеет высокую стоимость для доступа ко всем функциям. 
    \item
        MySql и MariaDB. Одни из самых известных баз данных, ныне MySql принадлежит Oracle, MariaDB является ответвлением от MySQL с открытым исходным кодом. Поддерживает шифрование, множество функций (например храенение координат и выполнение запросов данных о местоположении), высоко производительна, стабильна, работает во множестве известных операционных системах. MariaDB предназначена для совместимости с MySQL, однако существует проблема с сопоставлениями версий, а также инженеры MySQL вводят некоторые платную функциональность, недоступную MariaDB. \cite{web:drach}
    \item 
        Postgresql - не просто реляционная СУБД, а объектно-реляционная СУБД. Поддерживает вложенные и составные конструкции, которые не поддерживаются стандартными СУБД. Поддерживает обширный список типов данных, которые поддерживает PostgreSQL: uuid, денежный, перечисляемый, геометрический, бинарный, адресный (сети) и т.д. \cite{web:postgreshabr}
\end{itemize}

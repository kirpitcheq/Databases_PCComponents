\chapter{Конструкторский раздел}

\section{Формализация данных}
\subsection{Сущности}
%логично сначала описать атрибуты сущностей, а затем сделать вывод о том, что множество компонентов обладают одинаковыми атрибутами, поэтому можно выделить общие сущности, к которой будут связаны множество других

Перечислим компоненты
\begin{itemize}
  \item 
    Оперативная память
  \item 
    Процессор
  \item 
    Графическая карта
  \item 
    Графический процессор
  \item
    Жесткие диски (HDD)
  \item 
    Твердотельные диски (SSD)
  \item 
    Твердотельные диски (SSD) формата m2
  \item 
    Материнская плата
\end{itemize}
Все сущности являются компонентами. Некоторые компоненты являются подмножеством других, например множество энергонезависимой памяти

Таким образом образуется иерархия сущностей с общими атрибутами.

\subsection{Иерархия}
Выделение сущностей отобразим в формате иерархии, вложенные элементы в которой предполагаеют наличие общих атрибутов:
\begin{itemize}
  \item
    Страны
  \item
    Производители
  \item 
    Компоненты
    \begin{itemize}
      \item 
        Оперативная память
      \item 
        Процессор
      \item 
        Графическая карта
        \begin{itemize}
          \item 
            Графический процессор
        \end{itemize}
      \item 
        Энергонезависимая память
        \begin{itemize}
          \item 
            Жесткие диски (HDD)
          \item 
            Твердотельные диски (SSD)
            \begin{itemize}
              \item 
                Твердотельные диски (SSD) формата m2
            \end{itemize}
        \end{itemize}
      \item 
        Материнская плата
    \end{itemize}
\end{itemize}
Общие сущности при этом служат также и для объединения всех компонентов и предоставления общей информации о них. 

Отобразим все атрибуты и связи сущностей в диаграмме сущность-связь (ERD) (рисунок \ref{img:erd})

\includeimage
{erd} % Имя файла без расширения (файл должен быть расположен в директории inc/img/)
{f} % Обтекание (без обтекания)
{h} % Положение рисунка (см. figure из пакета float)
{\textwidth} % Ширина рисунка
{ERD диаграмма} % Подпись рисунка

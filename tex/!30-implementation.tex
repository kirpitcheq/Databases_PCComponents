\chapter{Технологический раздел}

\section{Выбор СУБД}
Сформируем требования, предъявляемые к СУБД:
\begin{itemize}
	\item 
Одной из самых эффективных, распространенных и надёжных операционных систем является ОС Linux, поэтому СУБД должно работать в такой ОС
	\item 
В предыдущем разделе представлено множество сущностей, обладающих общими атрибутами, для которых требуется сформировать связь таблиц нескольких сущностей к одной. Можно сформировать для каждой из вложенных таблиц внешний ключ ссылающийся на одну главную таблицу, что потребует как дополнительный атрибут идентификатор, так и формирование для каждой записи вложенной таблицы, запись в главной таблице, а затем указание идентификатора на эту запись. Таким образом происходит усложнение работы как при формировании таблиц для сущностей, так и для обработки данных особенно при заполнении различных таблиц. Система должна иметь механизм работы, позволяющий облегчить процесс создания таблиц с частью одинаковых атрибутов
\end{itemize}

Была выбрана СУБД postgresql, в которой:
\begin{itemize}
	\item 
		встроена возможность наследования таблиц, которая позволяет одним ключевым словом с указанием таблицы родителя INHERITES(tablename) избежать дублирования атрибутов и организовать общую таблицу с множеством записей, имеющая общий первичный ключ для нескольких таблиц с одинаковой одной частью атрибутов и различной другой.
	\item 
		поддержка ОС Linux
	\item 
		бесплатное распространение ПО
\end{itemize}
Для работы с исходными текстами модулей был выбран следующий набор программного обеспечения:
\begin{itemize}
	\item терминальный текстовый редактор VIM
	\item ПО с графическим интерфейсом pgAdmin
\end{itemize}

\section{Создание объектов базы данных}
\subsection{Таблицы}
В листинге \ref{lst:create_tables.sql} скрипт для создания и настройки таблиц для СУБД PostgreSQL

\listingminted{create_tables.sql}{sql}
{Скрипт создания и настройки таблиц на языке PL\/pgSQL}

\subsection{Функции}
Из предыдущего раздела видно, что количество атрибутов в таблицах относительно велико. Перед тем как добавлять новую запись, требуется подготовить данные в других вспомогательных таблицах, в которых запись может существовать или же отсутствовать. Сформируем вспомогательные функции для упрощения создания новых данных
\begin{enumerate}
	\item 
		для облегчения добавления элементов в промежуточные таблицы, которые имеют простую структуру:
		\begin{itemize}
			\item 
				идентификатор
			\item
				имя
		\end{itemize}
		создадим функцию \verb|find_or_create_el(_tab,namevalue)|, которая ищет элемент по указанному значению атрибута (имени) и если не находит, то создает такой элемент и в любом случае возвращает идентификатор
	\item
		\verb|find_or_create_manufacturer| для условного добавления производителя по названию и стране. Функция является комбинацией использования предыдущей функции для двух таблиц Countries и Manufacturers
	\item
		\verb|find_or_create_manuf| условное добавления конкретной категории производителя (процессоров, оперативной памяти, графических карт и т.д.) по идентификатору производителя таблицы Manufacturers и названию таблицы категории производителя (CPUManufacturers, RAMManufacturers и т.д.)
	\item
		\verb|add_cpusocket| условное добавления сокета процессора по названию производителя, страны производителя и названию сокета
	\item
		\verb|add_gpu| добавление записи характеристик графического процессора в таблицу GPUs перед добавлением записей характеристик графических карт в таблицу GraphicCards
\end{enumerate}
Описание функций представлено в листинге \ref{lst:functions.sql}

\listingminted{functions.sql}{sql}
{Скрипт создания функций на языке PL\/pgSQL}

\makeunnumchapt{ВВЕДЕНИЕ}

В современном мире информационных технологий базы данных играют ключевую роль в организации и хранении данных. Такие системы в отличии от традиционных способов хранения, например в обычном текстовом файле, имеют ряд преимуществ: 
\begin{itemize}
  \item 
  компактное хранение
  \item 
  быстрый доступ к информации
  \item 
  изменение информации для множества записей
  \item 
  сохранность информации
  \item 
  доступность информации
  и т.д.
\end{itemize}
Однако, чтобы с такой информацией можно было работать, требуется предварительно спроектировать базу данных, применяя всевозможные решения в зависимости от требований, предъявляемых к системе, используя:
\begin{itemize}
  \item 
  заранее подготовленные функции для работы с данными в этой системе
  \item 
  разграничение прав доступа для групп пользователей базой данных для защиты данных
  и т.д.
\end{itemize}

От того как качественно спроектирована система будет зависеть время доступа к данным, время на дополнение базы данных новыми записями, время на изменение имеющейся в базе данных информации.

Для каждого типа информации требуется свой подход к проектированию, т.к. та или иная группа информации может иметь схожие признаки, которые разумно выделить, рапределить в отдельные сущности, а какие то объединить.

В данном курсовом проекте мы рассмотрим основные принципы проектирования баз данных, специфичные для отрасли информационных технологий, и разработаем структуру базы данных, позволяющую эффективно хранить и управлять информацией о комплектующих персонального компьютера.

Целью курсовой работы является проектирование и разработка базы данных характеристик комплектующих персонального стационарного комьютера.
